\documentclass[12pt]{article}
\usepackage{fancyhdr}
\usepackage{amsmath,amsfonts,enumerate}
\usepackage{color,graphicx}
\pagestyle{fancy}
%%%%%%%%%%%%%%%%%%%%%%%%%%%%%%%%%%%%%%%%%%%%%%%%%
% Do your customization here
%%%%%%%%%%%%%%%%%%%%%%%%%%%%%%%%%%%%%%%%%%%%%%%%%
\newcommand{\masunitnumber}{MTH 213}
\newcommand{\examdate}{December 2011}
\newcommand{\academicyear}{2011-2012}
\newcommand{\semester}{I}
\newcommand{\coursename}{Experimental Mathematics}
\newcommand{\numberofhours}{2}

\newcommand{\ZZ}{\mathbb{Z}}
\newcommand{\CC}{\mathbb{C}}
\newcommand{\RR}{\mathbb{R}}
\newcommand{\FF}{\mathbb{F}}
%\DeclareMathOperator{\diam}{diam}
%%%%%%%%%%%%%%%%%%%%%%%%%%%%%%%%%%%%%%%%%%%%%%%%%
% Don't touch anything from here till instructions
% to candidates
%%%%%%%%%%%%%%%%%%%%%%%%%%%%%%%%%%%%%%%%%%%%%%%%%
\lhead{}
\rhead{}
\chead{{\bf NANYANG TECHNOLOGICAL UNIVERSITY}}
\lfoot{}
\rfoot{}
\cfoot{}
\begin{document}
\setlength{\headsep}{5truemm}
\setlength{\headheight}{14.5truemm}
\setlength{\voffset}{-0.45truein}
\renewcommand{\headrulewidth}{0.0pt}
\begin{center}
SEMESTER \semester\ EXAMINATION \academicyear
\end{center}
\begin{center}
{\bf \masunitnumber\ -- \coursename}
\end{center}
\vspace{20truemm}

\noindent \examdate\hspace{55truemm} TIME ALLOWED: \numberofhours\ HOURS

\vspace{19truemm}
\hrule
\vspace{19truemm}

\noindent\underline{INSTRUCTIONS TO CANDIDATES}
\vspace{8truemm}
%%%%%%%%%%%%%%%%%%%%%%%%%%%%%%%%%%%%%%%%%%%%%%%%%%%%%%
% Adjust your instructions here
%%%%%%%%%%%%%%%%%%%%%%%%%%%%%%%%%%%%%%%%%%%%%%%%%%%%%%
\begin{enumerate}
\item This examination paper contains {\bf FIVE (5)} questions and comprises 
{\bf FOUR (4)} printed pages.

\item Answer all questions. 
The marks for each question are indicated at the beginning of each question.


\item Answer each question beginning on a {\bf FRESH} page of the answer book.

\item This {\bf IS NOT an OPEN BOOK} exam.

\item Candidates may use calculators. However, they should write down systematically the steps in the workings.
\end{enumerate}

%%%%%%%%%%%%%%%%%%%%%%%%%%%%%%%%%%%%%%%%%%%%%%%%%
% leave this as it is
%%%%%%%%%%%%%%%%%%%%%%%%%%%%%%%%%%%%%%%%%%%%%%%%%
\newpage
\lhead{}
\rhead{\masunitnumber}
\chead{}
\lfoot{}
\cfoot{\thepage}
\rfoot{}
\setlength{\footskip}{45pt}
%%%%%%%%%%%%%%%%%%%%%%%%%%%%%%%%%%%%%%%%%%%%%%%%%%
% put your exam questions here
%%%%%%%%%%%%%%%%%%%%%%%%%%%%%%%%%%%%%%%%%%%%%%%%%%


\paragraph{Question 1.}\hfill (20 marks)\\
\begin{enumerate}[(i)]
    \item Write a function \verb|func1| that takes as input a list of
    3 numbers $[a_2, a_1, a_0]$ and returns the polynomial $a_2 x^2 + a_1 x
      + a_0$ which has these numbers as coefficients.\\
      For example, \verb|func1([1, 2, 3])|  should
      return \verb|x^2 + 2*x + 3|.
    \item Write a  function \verb|func2| which accepts a list $[a_{n-1},
    a_{n-2}, \dots, a_1, a_0]$ for any length $n$, and a number $k\geq 0$
    and returns the $k$-th derivative of the
polynomial $a_{n-1} x^{n-1} + a_{n-2} x^{n-2} + \cdots
    + a_1 x + a_0.$
\end{enumerate}


\bigskip
\bigskip
\bigskip
\paragraph{Question 2.}\hfill (20 marks)\\
Carol wants to compute the Taylor series of a function $f(x)$ around the
point $x=1$, up to the $n$-th term. The first term is clearly the value
$f(1)$. 
Carol implemented a function \verb|get_taylor_coeff| to compute the
Taylor coefficients. But this function is giving incorrect answers. Find
the error(s) in the function and correct the error(s).
\begin{verbatim}
# Get the taylor coefficients of the function f(x) upto the
# k-th term around the point x0. 
# For k=1, it should return [f(x0)].
def get_taylor_coeff(f, x0, k):
    coeffs = []
    for i in srange(1, k):
        t(x) = f.derivative(x, i)
        coeffs.append(t(i))
    return coeffs

# Example usage of get_taylor_coeff
f(x) = x^20 - x^4 + 1
get_taylor_coeff(f, 1, 3)
\end{verbatim}


\bigskip
\bigskip
\bigskip
\newpage 


\paragraph{Question 3.}\hfill (20 marks)\\
Consider the following interpolation problem. Let 
$$p(x) = a_{n-1} x^{n-1} +  a_{n-2} x^{n-2}
+ \cdots + a_1 x + a_0$$ be a polynomial. The graph of the corresponding
function $x\mapsto p(x)$ passes through the
points $(x_0, y_0), (x_1, y_1), \dots, (x_{n-1}, y_{n-1})$.
\noindent Adam wrote the following Sage code to return the list $[a_{n-1},\dots,a_0]$
of coefficients of the polynomial $p(x)$.
Here, \verb|ylist| is the list $[y_0, y_1, \dots, y_{n-1}]$ and
\verb|xlist| is the list $[x_0, x_1, \dots, x_{n-1}]$.
\begin{verbatim}
def get_coeff(xlist, ylist):
    n = len(xlist)
    def f(i, j):
        return xlist[i]^j
    M = matrix(RDF, n, n, f)
    yvec = vector(RDF, ylist)
    return M.solve_right(yvec)
\end{verbatim}
\begin{enumerate}[(i)]
    \item Adam is getting an incorrect answer   from \verb|get_coeff| when he
    is trying to get the coefficients of a degree two polynomial which
    passes through the points $(1, 5), (2, 10), (3, 17)$. Find the error(s) in
    the function \verb|get_coeff| due to which Adam is getting the
    incorrect answer. Give the corrections.
    \item Is there any degree 2 polynomial $p(x)$ for which the
    \emph{incorrect} function \verb|get_coeff| would still give a correct
    solution? If so, then give an example of such a polynomial. If such
    a polynomial cannot be obtained, explain why this is the case.
\end{enumerate}

\bigskip
\bigskip
\newpage 

\paragraph{Question 4.}\hfill (20 marks)\\
Eve decided to compute a certain function using recursion. The function Eve
wrote is the following: 
\begin{verbatim}
def compute(a, b):
    if a < 0 or b < 0 or a < b:
        return 0
    if b == 0:
        return 1
    return a*compute(a-1, b-1)/b
\end{verbatim}
\begin{enumerate}[(i)]
    \item What mathematical function is Eve's function \verb|compute()|
    evaluating?
    \item Write a non-recursive version of Eve's function \verb|compute()|.
\end{enumerate}
\bigskip
\bigskip
\bigskip
\bigskip

\paragraph{Question 5.}\hfill (20 marks)\\
Let $\pi$ be a permutation of the set $I=\{0,\dots,n-1\}$.  The {\em orbits} of
$\pi$ on $I$ are the  equivalence classes of the binary relation $\equiv_{\pi}$ on
$I$, so that $x\equiv_{\pi} y$ if and only if there exist $i\geq 0$ such that
$\pi^i(x)=y$. Here $\pi^i$ denotes the $i$-th iteration of $\pi$, i.e.
$\pi^0(x)=x$, 
$\pi^1(x)=\pi(x)$, $\pi^2(x)=\pi(\pi(x))$, etc.  Write a Sage function that
takes $\pi$ as a list of length $n$ of numbers in $I$ and returns the list of
lengths of the orbits of $\pi$ on $I$.

\vfill
\begin{center}{\bf END OF PAPER}\end{center}
\end{document}
