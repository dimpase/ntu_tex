\documentclass[twocolumn]{article}
\usepackage[top=2in, bottom=1in, left=0.5in, right=1in ]{geometry}
\usepackage{graphicx}
\usepackage{wallpaper}
\usepackage{amsfonts}
\usepackage{amssymb}
\usepackage{amsmath}
\usepackage{url}
\CenterWallPaper{1}{URECA_Poster_Template.pdf}
\newcommand{\sect}[1]{\vspace{-1mm} \paragraph{{\large {#1}.}}}
\begin{document}

\thispagestyle{empty}

\begin{huge}
\begin{center}
Shape Reconstruction from Moments
\end{center}
\end{huge}

\begin{large}

\sect{Introduction}
Inverse problems for measures supported on plane shapes (e.g. polygons) deal
with conversion of observations (more precisly, some moments of the
measure) into information about the shape (e.g. vertices of the polygon).
Complex moments are defined as 
$$m_k = \iint_\triangle{z^k}dx\,dy, \quad z\in \mathbb{C}, \quad k=0,1,2,\dots$$ 
This is quite useful in real life as there are lots of parameters which we
cannot measure directly. The advantage of using moments is that we can obtain
the properties of these parameters from the shape that we construct using
moments. The inverse problems have many applications in areas such as
computerized tomography, image processing, signal processing and inverse
potential theory. 

\sect{Objective}
One of our aims is to reconstruct plane triangles from the uniform measures on them, 
i.e. find a relation between
$z_1,z_2,z_3$ and $m_1,m_2,m_3$ where $z_1,z_2,z_3$ are vertices of a triangle.
We also try to formulate a moment generating function for polyhedron.

\sect{Methodology}
Let $f(z)=(z-z_1)(z-z_2)(z-z_3)=\sum_{k=0}^{3}{a_kz^k}$.
In \cite{davis77} one finds the following formulas
$$z_1+z_2+z_3=\frac{3m_1}{m_0}$$ $$z_1z_2+z_1z_3+z_2z_3=\frac{9m_1^2 -6m_0m_2}{m_0^2}$$ $$z_1z_2z_3=\dfrac{10m_0^2m_3+27m_1^3-36m_0m_1m_2}
{m_0^3}$$

Notice that the three complex numbers $z_1,z_2,z_3$ depend on first four
moments. But we can reduce the dependency to only three moments by finding the
relation between $m_0$ and $m_1,m_2,m_3$ through a formula for the area 
$m_0=m_0(z_1,z_2,z_3)$ of the triangle  
$$m_0=\frac{i}{4}\det(V), \quad V=
\begin{pmatrix}
z_1 & \overline{z_1} & 1 \\
z_2 & \overline{z_2} & 1 \\
z_3 & \overline{z_3} & 1 \\
\end{pmatrix}.$$
By expanding $VV^T$, we can obtain an equation which relates $m_0$ and $a_1,a_2,a_3$.

In \cite{PS12}, the moment generating function 
$$\Psi_\mu (u)=\sum_{j=0}^{\infty}{\frac{(j+2)(j+1)}{2}{m_j(\mu)u^j}}=\int \frac{d\mu(z)}{(1-uz)^3}$$
where moments are coefficients, is obtained. Thus, by using the moments
given, we can obtain try to recover the shape (e.g. the vertices of the triangle). 
We try to extend this to higher-dimensional
polyhedra, i.e. to design a moment generating function which gives us
the vertices of the polyhedron from moments given.This is quite useful because
we can convert the power series into closed form and hence obtain its
vertices.  Besides, we try to test the uniqueness of the polyhedron from
series obtained, that is, how many polyhedrons can be reconstructed from the
series obtained.

\begin{thebibliography}{10}
\bibitem{wikiinverse} \url{http://en.wikipedia.org/wiki/Inverse_problem}
\bibitem{davis77} P.J. Davis, Plane regions determined by complex moments. 
J. of Approximation Theory, 19(1977) 
148-153
\bibitem{PS12} D. Pasechnik, B. Shapiro. 
On polygonal measures with vanishing harmonic moments. arXiv:1209.4014 (2012)
\end{thebibliography}
\end{large}
\end{document}
