\documentclass[12pt]{article}
\usepackage{fancyhdr,listings,amsfonts,amsmath,color}
\pagestyle{fancy}
\parindent0em
%%%%%%%%%%%%%%%%%%%%%%%%%%%%%%%%%%%%%%%%%%%%%%%%%
% Do your customization here
%%%%%%%%%%%%%%%%%%%%%%%%%%%%%%%%%%%%%%%%%%%%%%%%%
\newcommand{\masunitnumber}{MAS 722}
\newcommand{\examdate}{December 2012}
\newcommand{\academicyear}{2012-2013}
\newcommand{\semester}{I}
\newcommand{\coursename}{Topics in Pure Mathematics I}
\newcommand{\numberofhours}{3}
\newcommand{\Z}{\mathbb{Z}}
\newcommand{\F}{\mathbb{F}}
\newcommand{\C}{\mathbb{C}}
\newcommand{\Q}{\mathbb{Q}}
\newcommand{\PP}{\mathbb{P}}
\newcommand{\R}{\mathbb{R}}
\newcommand{\GL}{\mathrm{GL}}
\definecolor{listinggray}{gray}{0.9} \lstset{language=c++}
\lstset{backgroundcolor=\color{listinggray},rulecolor=\color{blue}}
\lstset{linewidth=\textwidth} \lstset{commentstyle=\tt,
stringstyle=\ttfamily,showspaces=false,basicstyle=\tt}
\lstset{showstringspaces=false, keepspaces=true}
\newcommand{\code}[1]{\lstinline@#1@}
\newcommand{\Code}[1]{\colorbox{listinggray}{\lstinline@#1@}}
%%%%%%%%%%%%%%%%%%%%%%%%%%%%%%%%%%%%%%%%%%%%%%%%%
% Don't touch anything from here till instructions
% to candidates
%%%%%%%%%%%%%%%%%%%%%%%%%%%%%%%%%%%%%%%%%%%%%%%%%
\lhead{}
\rhead{}
\chead{{\bf NANYANG TECHNOLOGICAL UNIVERSITY}}
\lfoot{}
\rfoot{}
\cfoot{}
\begin{document}
\setlength{\headsep}{7truemm}
\setlength{\headheight}{14.5truemm}
\setlength{\voffset}{-0.4truein}
\renewcommand{\headrulewidth}{0.0pt}
\begin{center}
\bf SEMESTER \semester\ EXAMINATION \academicyear
\end{center}
\begin{center}
{\bf \masunitnumber -- \coursename}
\end{center}
\vspace{20truemm}

\noindent \examdate\hspace{55truemm} TIME ALLOWED: \numberofhours\ HOURS

\vspace{10truemm}
\hrule
\vspace{10truemm}

\noindent\underline{INSTRUCTIONS TO CANDIDATES}
\vspace{5mm}
%%%%%%%%%%%%%%%%%%%%%%%%%%%%%%%%%%%%%%%%%%%%%%%%%%%%%%
% Adjust your instructions here
%%%%%%%%%%%%%%%%%%%%%%%%%%%%%%%%%%%%%%%%%%%%%%%%%%%%%%
\begin{enumerate}
\item  This examination paper contains {\bf FIVE (5)} questions and
comprises {\bf THREE (3)} printed pages.\\
\item
Answer all questions. \\
\item  Answer each question beginning on a {\bf FRESH} page of the answer
book.\\
\item Candidates may use calculators.
However, they should write down systematically the steps in the workings.
\item This is an {\bf open book} examination.
\end{enumerate}

%%%%%%%%%%%%%%%%%%%%%%%%%%%%%%%%%%%%%%%%%%%%%%%%%
% leave this as it is
%%%%%%%%%%%%%%%%%%%%%%%%%%%%%%%%%%%%%%%%%%%%%%%%%
\newpage
\lhead{}
\rhead{\masunitnumber}
\chead{}
\lfoot{}
\cfoot{\thepage}
\rfoot{}
\setlength{\footskip}{45pt}
\newcommand{\Ques}[1]{\paragraph{\underline{Question {#1}}}\hfill\\\vspace{3mm}\\}
%%%%%%%%%%%%%%%%%%%%%%%%%%%%%%%%%%%%%%%%%%%%%%%%%%
% put your exam questions here
%%%%%%%%%%%%%%%%%%%%%%%%%%%%%%%%%%%%%%%%%%%%%%%%%%

%%%%%%%%%%%%%%%%%%%%%%%%%%%%%%%%%%%%%%%%%%%%%%%%%%%%%%%%%%%%%%%%%%%%%%%%%%%%%%%%%%%%%%
\Ques{ 1}
Let $Z$ be a Zariski-closed set in $A^n$, the $n$-dimensional affine space over a
field $\F$. 
\begin{itemize}
\item[(i)] 
Show that $Z$ is irreducible if and only if the coordinate ring $\F[Z]$ has no
zero divisors. \hfill (10~marks)
\item[(ii)] 
Give an example, with justification, of an irreducible $Z$.
 \hfill (5~marks)
\end{itemize}

\hspace*{2cm} \hfill Total: 15~marks
\bigskip
\bigskip
\bigskip
\bigskip

\Ques{ 2}
Let $N$ be a subvariety (i.e. an irreducible Zariski closed subset) 
of an affine algebraic variety $M\subset A^n$.
Show that $\dim N\leq \dim M$.

\hspace*{2cm} \hfill Total: 20~marks
\bigskip
\bigskip
\bigskip
\bigskip

\Ques{ 3}
Let $f:=f(x,y)$ be an irreducible complex cubic. 
\begin{itemize}
\item[(i)] 
Show that the curve $f=0$ has at most one singular point, 
and that the multiplicity of this point is two.
\hfill (10~marks)
\\
({\bf Hint}. Use the Weierstrass normal form for the cubics.)
\item[(ii)] 
Show that the curve $f=0$ is either isomorphic to the nodal cubic $y^2=x^3+x^2$, or 
isomorphic to the cuspidal cubic $y^2=x^3$.
 \hfill (15~marks)
\end{itemize}

\hspace*{2cm} 
\hfill Total: 25~marks
\bigskip
\bigskip
\bigskip
\newpage

\Ques{ 4}
A generating set of an ideal $I\subset\F[x_1,\dots,x_n]$ is called \emph{universal Gr\"{o}bner basis}
if it is a  Gr\"{o}bner basis of $I$ with respect to any term order. 
Compute a universal Gr\"{o}bner basis of the ideal $(x-y^2,xy-x)\subset\C[x,y]$.

 \hfill Total: 20~marks
\bigskip
\bigskip
\bigskip

\Ques{ 5}
Let $S$ be the set of polynomials of the form $g^\ell$, for each 
homogeneous $g\in\C[x_0,\dots,x_n]$ of degree $k$. Identify $S$ with a closed subset
of the Veronese variety in $\PP^{\binom{k\ell+n}{n}}$.

\hspace*{2cm} 
\hfill Total: 20~marks



\vfill
\centerline{\bf END OF PAPER}
\end{document}
